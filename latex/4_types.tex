\section{Types}
\subsection{Description}
\begin{enumerate}
    \item When the variable may contains specific values (0 - off, 1 - on) new type should be created.
    \item Never typedef arrays and pointers.
    \item Enum, structure, union should create new types.
    \item Always use type depending on your job.
\end{enumerate}

\subsection{Examples}
\begin{enumerate}
    \item 
\begin{lstlisting}[language=C,style=C99]
/* Correct. This type tells exactly what this variable represents */
typedef enum parser_state_t
{
    PARSER_STATE_WORKING,
    PARSER_STATE_STOPPED,
} parser_state_t;

parser_state_t parser_state = PARSER_STATE_STOPPED;

parser_state = parse(...);

if (parser_state == PARSER_STATE_WORKING)
{
    do_this();
}
else
{
    do_this();
}

/* Incorrect */
uint8_t parser_state = 0;

parser_state = parse(...);

if (parser_state == 0)
{
    do_this();
}
else
{
    do_this();
}
\end{lstlisting}     
    
    \item 
\begin{lstlisting}[language=C,style=C99]
/* Correct. Typedef structures and basic types */
typedef uint8_t byte_t;
typedef Rbt Rbt;

/* Incorrect. Never do this ! */
typedef void* void_ptr;
typedef int int_array[CONSTANT];
\end{lstlisting}    
    
    \item 
\begin{lstlisting}[language=C,style=C99]
/* Correct */
typedef enum option
{
    OPTION_1,
    OPTION_2,
} option_t;

typedef struct Pair
{
    int first;
    int second;
} Pair;

/* Incorrect */
enum option
{
    OPTION_1,
    OPTION_2,
} option;

struct Pair
{
    int first;
    int second;
} Pair;

\end{lstlisting}    

    \item 
\begin{lstlisting}[language=C,style=C99]
/* Correct, if you need to represent byte use uint8_t from stdint.h */
typedef uint8_t byte_t;
typedef char byte_t; /* Possible */

/* Incorrect */
typedef uint8_t byte;
typedef char BYTE;


/* Correct, if you need to represent size or length use size_t from stddef.h */
const size_t array_size = sizeof(array) / sizeof(array[0]);

/* Incorrect */
const uint64_t array_size = sizeof(array) / sizeof(array[0]);


/* Correct, if you need to return size or length from function, but '-1' is error message use ssize_t from sys/types.h */
ssize_t darray_get_length(...);

/* Incorrect */
int64_t darray_get_length(...);


/* Correct, if you need to subtract two pointer, use ptrdiff_t from <stddef.h> */
int numbers[] = { /* lots of members */ };
 
int* p1_p = &numbers[18]; 
int* p2_p = &numbers[23];

const ptrdiff_t diff = p2 - p1;

/* Incorrect */
const int64_t diff = p2 - p1;
\end{lstlisting}
\end{enumerate}

\subsection{Reasons}
\begin{enumerate}
    \item It will make code more readable for other programmers.
    \item This practices makes code really hard to understand.
    \item By the new type we can represent new abstraction. It also helps to keep cohesion in code.
    \item For representing:
        \begin{itemize} 
            \item unsigned numbers - use unsigned types.
            \item signed numbers - use signed types.
            \item unsigned numbers with possible error code (-1) - use ssize\_t.
            \item floating point - use double, long double types with proper precision.
            \item etc.
        \end{itemize}
\end{enumerate}