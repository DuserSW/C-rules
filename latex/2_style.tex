\section{Style}
Clang-format is a tool to format C code according to a set of rules and heuristics. Like most tools, it is not perfect nor covers every single case, but it is good enough to be helpful. Clang-format is specially useful when moving code around and aligning/sorting. Help you follow the coding style rules, specially useful for those new in this project or working at the same time in several projects with different coding styles.

\subsection{Indentation}
Only spaces should be used. Indentation should be set to 4 spaces.

\subsection{Breaking long lines and strings}
Long line is defined as line with more than 140. Every long line should be split into more lines using some rules.
\subsubsection{Description}
\begin{enumerate}
    \item Break line after operator.
    \item Do not break structure chain. If structure chain will be longer than 140 chars, split it and rewrite it as pointers.
    \item Breaking function parameters should be aligned to each other.
\end{enumerate}

\subsubsection{Examples}
\begin{enumerate}
    \item 
\begin{lstlisting}[language=C,style=C99]
/* Correct. Break after an operator */
if (condition &&
    condition)
{
    /* some code */
}

/* Incorrect if line with conditions is longer than 140 characters */
if (condition && condition)
{
    /* some code */
}
\end{lstlisting}    

    \item 
\begin{lstlisting}[language=C,style=C99]
/* Correct */
a_p->b_p->c.d_p->g.f;

/* Correct */
ptr_p = a_p->b_p;
ptr_p->c.d_p->g.f;

/* Incorrect */
a_p->b_p->c.
    d_p->g.f;
\end{lstlisting} 

    \item 
\begin{lstlisting}[language=C,style=C99]
/* Correct. Break after comma. Function parameters should be aligned */
void foo(arg1,
         arg2,
         arg3);
         
/* Incorrect if argument list is longer than 140 characters */
void foo(arg1, arg2, arg3);
\end{lstlisting}  
\end{enumerate}

\subsubsection{Reasons}
\begin{enumerate}
    \item This style makes code more readable.
    \item Broken structure chain makes code hard to read and understand.
    \item Squashing code should be avoided. Aligned function arguments makes easier to analyze.
\end{enumerate}

\subsection{Placing braces and spaces}
\subsubsection{Description}
\begin{enumerate}
    \item Always after C keyword (if, switch, case, for, do, while), before and after operator (+, -, *, /), before comma space should be inserted. Exception is sizeof.. 
    \item Structure operators (->, .) and array operator([ ]) should be used without spaces between them.
    \item Braces should not be avoided in 1-line statements.
    \item Braces should be always used in switch case.
\end{enumerate}

\subsubsection{Examples}
\begin{enumerate}
    \item 
\begin{lstlisting}[language=C,style=C99]
/* Correct */
for (size_t i = 0; i < ARRAY_SIZE(array); ++i)
{
    printf("%4d", array[i]);
}

/* Incorrect */
for(size_t i = 0; i < ARRAY_SIZE(array); ++i)
{
    printf("%4d", array[i]);
}

/* Correct */
if (condition)
{
    do_this();
}

/* Incorrect */
if(condition)
{
    do_this();
}

/* Correct */
a + b     /* addition operator */
a - b     /* subtraction operator */
a / b     /* division operator */
a % b     /* modulus operator */
a ^ b     /* bitwise xor operator */
a & b     /* bitwise and operator */
a | b     /* bitwise or operator */
a < b     /* less than operator */
a <= b    /* less than or equal to operator */
a > b     /* greater than operator */
a >= b    /* greater than or equal to operator */
a == b    /* equal to operator */
a != b    /* not equal to operator */
a && b    /* logical and operator */
a || b    /* logical or operator */
a << b    /* shift left operator */
a >> b    /* shift right operator */
a = b     /* assignment operator */
a ? b : c /* ternary operator */
\end{lstlisting}  

    \item 
\begin{lstlisting}[language=C,style=C99]
/* Correct */
matrix[5][5] = 12;

/* Incorrect */
matrix [5] [5] = 12; 

/* Correct */
list_p->head_p = NULL;

/* Incorrect */
list_p -> head_p = NULL;
    
/* Correct */
t[]       /* array operator */
t()       /* function call operator*/
t.a       /* member operator */
t->a      /* arrow operator */
*t        /* dereference operator */
&t        /* address operator */
~t        /* bitwise negation operator (unary) */
!t        /* logic negation operator */
++t       /* preincrementation operator */
t++       /* postincrementation operator */
--t       /* predecrementation operator */
t--       /* postdecrementation operator */
sizeof(t) /* sizeof operator */
\end{lstlisting} 
    
    \item 
\begin{lstlisting}[language=C,style=C99]
/* Correct */
for (size_t i = 0; i < ARRAY_SIZE(array); ++i)
{
    printf("%4d", array[i]);
}

/* Incorrect */
for (size_t i = 0; i < ARRAY_SIZE(array); ++i)
    printf("%4d", array[i]);

/* Correct */
if (condition)
{
    do_this();
}

/* Incorrect */
if (condition)
    do_this();
\end{lstlisting}      

    \item 
\begin{lstlisting}[language=C,style=C99]
/* Correct */
switch (val)
{
    case 1:
    {
        do_this();
        calculate();
        break;
    }
    case 2:
    {
        /* some code */
    }
    default:
    {
        ERROR("something bad has happened\n");
        break;
    }
}

/* Incorrect */
switch (val)
{
    case 1:
        do_this();
        calculate();
        break;
    case 2:
        foo();
        break;
    default:
        ERROR("something bad has happened\n");
        break;
}
\end{lstlisting} 

\end{enumerate}

\subsubsection{Reasons}
\begin{enumerate}
    \item It is easier to read special language keywords and operator with space before them.
    \item It looks much better when array and structure operators are one sequence of characters.
    \item Code has to be readable that is why braces cannot be avoided in one line statements. Instead of it, ternary operator can be use.
    \item It allows to dismiss mistakes and allocate memory on stack if it is needed.
\end{enumerate}

% Bez przykladu bo bedzie clang-format
\subsection{White characters}
All trailing whitespaces should be deleted before commit.

% Zbyt proste na przyklad wiec nie ma przykladu
\subsection{Commenting}
Do not use // as comment, even if comment has only 1 line. This is cpp specific. C comment operator should be used: /* */. To comment documentation, doxygen comment operator should be used: /** */

\subsection{Casting}

\subsubsection{Description}

\begin{enumerate}
    \item Pointer to void should be never cast to and from
    \item All compatible and incompatible types should be cast
    \item Not used function parameters should be cast to void
    \item Not used returned value where value describes error of function should be cast to void
    \item Not used returned value from system functions should be always cast to void
\end{enumerate}

\subsubsection{Examples}

\begin{enumerate}
    \item 
\begin{lstlisting}[language=C,style=C99]
/* Correct */
int c = 0;
void* vptr_p = &c;

/* Incorrect */
int b = 5;
void* vptr_p = (void *)&b;

/* Incorrect */
int a = 3;
int* ptr_p = &a;
void* vptr_p = (void *)ptr;
\end{lstlisting}

    \item 
\begin{lstlisting}[language=C,style=C99]
/* Correct */
char a = '0';
int8_t b = (int8_t)a;

/* Incorrect */
char a = '0';
int8_t b = a;

/* Correct */
int c = 2;
long d = (long)c;

/* Correct */
int c = 2;
long d = c;
\end{lstlisting}

    \item 
\begin{lstlisting}[language=C,style=C99]
/* Correct */
void foo(int* a_p, int b, int c)
{
    /* c in no longer needed */
    (void)c;
    
    /* some code */
}

/* Incorrect */
void foo(int* a_p, int b, int c)
{
    /* c in no longer needed but not case to void */
    
    /* some code */
}
\end{lstlisting}

    \item 
\begin{lstlisting}[language=C,style=C99]
/* Correct */
(void)strncpy(dst, src, size_of);

/* Incorrect because printf returns number of 
characters , not error */
(void)printf("Hello world\n") ;
\end{lstlisting}
    
    \item 
\begin{lstlisting}[language=C,style=C99]
/* Correct because write makes syscall and returns number of bytes written, which can be used to check success or -1 if failure */
(void)write(fd, &buf, strlen(buf));

/* Incorrect */
write(fd, &buf, strlen(buf));
\end{lstlisting}
\end{enumerate}

\subsubsection{Reason}
\begin{enumerate}
    \item This implicit conversion is provided by compiler.
    \item It makes code longer but easier to read.
    \item This information tells clearly to programmer that input argument is no longer needed.
    \item As above
    \item As above
\end{enumerate}

\subsection{Allocating memory}
\subsubsection{Descripton}
\begin{enumerate}
    \item Do not use malloc to allocate memory. Instead of it, use calloc.
    \item Structure allocated on stack should be always reset.
    \item Do not use alloca to dynamically allocate memory on stack. C99 supports VLA and VLA should be used instead.
\end{enumerate}

\subsubsection{Examples}
\begin{enumerate}
    \item 
\begin{lstlisting}[language=C,style=C99]
/* Correct */
int64_t* ptr_p = calloc(5, sizeof(*ptr_p));

/* Incorrect */
int64_t* ptr2_p = malloc(sizeof(*ptr2_p) * 5);
\end{lstlisting} 

    \item 
\begin{lstlisting}[language=C,style=C99]
/* Correct. Always initialize arrays on stack */ 
void foo(void)
{
    uint32_t array[10] = {0};
    
    /* some code */
}

/* Incorrect */ 
void foo(void)
{
    uint32_t array[10]; /* Not initialized array */
    
    /* some code */
}

/* Correct. Always memset VLA because VLA cannot be initialize as arrays of constant known size */
void vla(size_t size)
{
    int vla_array[size];
    (void)memset(&vla_array[0],
                 0,
                 sizeof(vla_array));
    
    /* some code */
}

/* Incorrect */
void vla(size_t size)
{
    int vla_array[size]; /* Not initialized VLA */
    
    /* some code */
}
\end{lstlisting}

    \item 
\begin{lstlisting}[language=C,style=C99]
/* Correct */
void foo(const size_t size)
{
    /* Do not forget to memset VLA */
    int64_t array[size];
    (void)memset(&vla_array[0],
                 0,
                 sizeof(vla_array));
    
    /* some code */
}

/* Incorrect. Do not use alloca to allocating memory on stack. 
   Instead of it, use VLA, which is supported by C99 */
void foo(const size_t size)
{
    int64_t* ptr_p = alloca(sizeof(*ptr_p) * size);
    
    /* some code */
}
\end{lstlisting}
\end{enumerate}

\subsubsection{Reason}
\begin{enumerate}
    \item Always use calloc which set memory to zero. This would avoid unnecessary mistakes in future.
    \item Tis is often mistake. Dereference of memory allocated on stack, which has not been reseted. Always reset memory allocated on stack.
    \item Try to avoid obsolete alloca for dynamic allocations on stack. Instead of it, use feature from C99, which is variable length array.
\end{enumerate}